\documentclass[sigconf]{acmart}

\AtBeginDocument{ \providecommand\BibTeX{ Bib\TeX } }
\setcopyright{acmlicensed}
\copyrightyear{2025}
\acmYear{2025}
\acmDOI{XXXXXXX.XXXXXXX}

\acmConference[BI 2025]{Business Intelligence}{-}{-}

\begin{document}

\title{BI2025 Experiment Report - Group 009}
%% ---Authors: Dynamically added ---

          \author{Liliana Sulyok}
          \authornote{Student A, Matr.Nr.: 12307565}
          \affiliation{
            \institution{TU Wien}
            \country{Austria}
          }
          
          \author{Zita Dorina Marton}
          \authornote{Student B, Matr.Nr.: 12340246}
          \affiliation{
            \institution{TU Wien}
            \country{Austria}
          }
          

\begin{abstract}
  This report documents the machine learning experiment for Group 009, following the CRISP-DM process model.
\end{abstract}

\ccsdesc[500]{Computing methodologies~Machine learning}
\keywords{CRISP-DM, Provenance, Knowledge Graph, Machine Learning}

\maketitle

%% --- 1. Business Understanding ---
\section{Business Understanding}

\subsection{Data Source and Scenario}
The dataset is the Online Shoppers Purchasing Intention Dataset from the 
UCI Machine Learning Repository. It contains 12,330 anonymized, unique user
sessions from an e-commerce website collected over one year. The detaset 
includes behavioral, technical, and user-related attributes such as page
duration, bounce rates, visitor type. The binary target variable indicates 
whether a session resulted in revenue. 

Business Scenario:
We assume the role of data scientists for a mid-sized online retail company
that seeks to improve its marketing efficiency by identifing sessions with high
purchase intent in real time. This allows the marketing team to apply targeted
interventions (such as live chat support or limited-time discount pop-ups 
specifically for users likely to purchase) only when beneficial. 
Predicting purchase likelihood enables cost-efficient personalization, 
optimizing marketing resources and strategic decision-making

\subsection{Business Objectives}
Primary Objective:
Increase online revenue by identifying sessions with high purchase intent and 
by proactively engaging customers using targeted, cost-efficient interventions 
that increase the probability of conversion.

Secondary Objectives:
Improve the overall conversion rate through real-time personalization.
Reduce unnecessary promotional costs on users who are highly unlikely to buy
regardless of incentives.

%% --- 2. Data Understanding ---
\section{Data Understanding}
\textbf{Dataset Description:} Describe the data set.

The following features were identified in the dataset:

\begin{table}[h]
  \caption{Raw Data Features}
  \label{tab:features}
  \begin{tabular}{lp{0.2\linewidth}p{0.4\linewidth}}
    \toprule
    \textbf{Feature Name} & \textbf{Data Type} & \textbf{Description} \\
    \midrule
    Administrative & integer> & Number of administrative pages visited. \\
    Administrative\_Duration & double> & Time spent on administrative pages (sec). \\
    BounceRates & double> & Average bounce rate. \\
    Browser & integer> & Browser code. \\
    ExitRates & double> & Average exit rate. \\
    Informational & integer> & Number of informational pages visited. \\
    Informational\_Duration & double> & Time spent on informational pages (sec). \\
    Month & string> & Month of session. \\
    OperatingSystems & integer> & Operating system code. \\
    PageValues & double> & Average value a user visited before completing a transaction. \\
    ProductRelated & integer> & Number of product-related pages viewed. \\
    ProductRelated\_Duration & double> & Total time on product pages (sec). \\
    Region & integer> & Region code. \\
    Revenue & boolean> & Class label: whether purchase occurred. \\
    SpecialDay & double> & Proximity to a special day (0–1). \\
    TrafficType & integer> & Traffic source category. \\
    VisitorType & string> & Returning or new visitor. \\
    Weekend & boolean> & Boolean indicating weekend visit. \\
    \bottomrule
  \end{tabular}
\end{table}

%% --- 3. Data Preparation ---
\section{Data Preparation}
\subsection{Data Cleaning}
Describe your Data preparation steps here and include respective graph data.


%% --- 4. Modeling ---
\section{Modeling}

\subsection{Hyperparameter Configuration}
The model was trained using the following hyperparameter settings:

\begin{table}[h]
  \caption{Hyperparameter Settings}
  \label{tab:hyperparams}
  \begin{tabular}{lp{0.4\linewidth}l}
    \toprule
    \textbf{Parameter} & \textbf{Description} & \textbf{Value} \\
    \midrule
    
    \bottomrule
  \end{tabular}
\end{table}

\subsection{Training Run}
A training run was executed with the following characteristics:
\begin{itemize}
    \item \textbf{Algorithm:} 
    \item \textbf{Start Time:} 
    \item \textbf{End Time:} 
    \item \textbf{Result:}  = 
\end{itemize}

%% --- 5. Evaluation ---
\section{Evaluation}

%% --- 6. Deployment ---
\section{Deployment}

\section{Conclusion}

\end{document}
